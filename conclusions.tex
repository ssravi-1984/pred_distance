%\section{Conclusions and Future Research Directions}
%%\vspace*{-0.3in}
\section{Summary and Future Research Directions}
\label{sec:concl}
%%\vspace*{-.22in}

%%\textcolor{red}{(To be added.)} 
We investigated questions regarding the time evolution of
similar configurations.
We presented examples to show that SyDSs may exhibit
extreme behaviors in the evolution of configurations. 
In particular, our examples showed that in SyDSs, one may have
highly dissimilar configurations whose predecessors are similar.
Likewise, one may have very similar configurations whose predecessors
are highly dissimilar.
We also showed that the problem of computing similarity
measures of predecessors of given configurations is, in general,
computationally intractable.
We presented experimental results on similarity measures using
some restricted forms of underlying graphs and local functions.

There are several directions for future work.
For example, it is of interest to formally analyze the
behavior of restricted classes of SyDSs.
It will also be useful to identify restricted classes of SyDSs
for which similarity measures can be computed or approximated efficiently.
Finally, instead of considering one step predecessors, one may consider 
similarity issues for $t$-step predecessors for $t \geq 2$.

\smallskip

\noindent
\textbf{Acknowledgments:}~
This work has been partially supported by
DARPA Cooperative Agreement D17AC00003 (NGS2),
DTRA CNIMS (Contract HDTRA1-11-D-0016-0001),
NSF DIBBS Grant ACI-1443054,
NSF BIG DATA Grant IIS-1633028 and
NSF EAGER Grant CMMI-1745207.

%% \genprob{} for other uniform SyDSs,
%% \genprob{} with constraints to model fixed point
%% existence, etc. }

