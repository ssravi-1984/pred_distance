\section{Introduction}
\label{sec:intro}

Discrete graph dynamical systems are
generalizations of cellular automata (CA) \cite{Wol-1987,Gut-1989}.
They serve as a useful formal model in may 
contexts, including  multi-agent systems, 
simulations of networked entities and propagation
of contagions in social networks 
(e.g., \cite{Woo-2002,MR-2007,Valente-1996}).
Here, we focus on one such class of graph dynamical systems,
namely \emph{synchronous} discrete dynamical systems (SyDSs).
Informally,
a SyDS\footnote{Formal definitions associated with SyDSs are presented
in Section~\ref{sec:prelim}.}
 consists of an undirected graph whose vertices represent
entities and edges represent local interactions among entities.
Each node $v$ has a Boolean state and
a local function $f_v$
whose inputs are the current state of $v$ and those its neighbors;
the output of $f_v$ is the next state of $v$.
The vector consisting of the state values of all the nodes at each time instant
is referred to as the \textbf{configuration} of the system at that instant.
In each time step, all nodes of a SyDS compute and update their
states \emph{synchronously}.
Starting from a (given) initial configuration,
the time evolution of a SyDS consists of a
sequence of successive configurations, which is also called a
\textbf{trajectory}.

In this paper, we examine questions related to the evolution
of configurations that are \emph{similar}. 
We measure the degree of similarity between two configurations
by their Hamming distance (i.e., the number of positions
where the two configurations differ). 
Thus, the smaller the Hamming distance between two configurations, 
the larger is the degree of similarity between them. 
The goal of our study is to obtain an
understanding of when similar configurations may evolve along 
similar trajectories.
As a concrete and simplified version of 
this general research question, we consider the
following problem: given two similar configurations, how similar are
their \emph{predecessors} (i.e., configurations that
just preceded the given configurations in 
the time evolution of the system)?
We study this problem both analytically and experimentally. 
Our results are summarized below.

%\smallskip

\noindent
\textbf{Summary of Results.}~ 
Our theoretical results are presented in Section~\ref{sec:analysis}.
They  show that SyDSs may exhibit
extreme behaviors with respect to the evolution of configurations.
For example, one of our results (Proposition~\ref{pro:close-close})  
shows that there are SyDSs in which
for any two configurations \calcone{} and \calctwo{} 
which differ in $h$ bits, there are respective predecessors 
$\calcone'$ and $\calctwo'$ which also differ in exactly $h$ bits.
We also show (Proposition~\ref{pro:close-far})
how to construct examples of SyDSs where two very similar 
configurations (which are 1-close) have highly dissimilar predecessors
(i.e., they differ in all the bits).
Further, we present examples of SyDSs (Proposition~\ref{pro:far-close})
in which highly dissimilar configurations have predecessors that are 1-close.
These examples point out that SyDSs may exhibit high level of sensitivity
with respect to initial conditions (i.e., starting configurations).
We also show that the problem of computing similarity
measures of the predecessors of two given configurations 
is, in general, computationally intractable.

Our experimental results presented in Section~\ref{sec:experiments}
are based on using public domain Boolean Satisfiability
(SAT) solvers (e.g., \cite{sat-live}).
We illustrate how the problem of computing predecessors of a
configuration can be expressed as an instance of the SAT
problem \cite{GJ-1979}.
Our experiments consider restricted classes of graphs and local functions.
\textcolor{red}{(More to be added after we have the experimental results.)}

\smallskip

\noindent
\textbf{Related Work.}~ %%\textcolor{red}{(To be revised)}
Computational problems associated with 
discrete dynamical systems 
have been addressed by many researchers.
For example, Barrett et al. \cite{BH+06} studied the
reachability problem under the sequential
update model, where a permutation of the vertices is also given,
and state updates are carried out in the order specified by the
permutation.  
%% They established computational intractability results
%% for the general case and obtained polynomial time algorithms when
%% local transition functions are limited to certain Boolean functions.
Bounds on the lengths of configuration cycles and transients 
(i.e., trajectories that end in configuration cycles) in restricted versions
of dynamical systems under the sequential update model are established
in \cite{MR-2007}.  
%% Wooldridge \cite{Woo-2002} presents a discussion
%% of complexity results for multi-agent systems.  
Tosic \cite{Tos-2010,Tosic-2017} presented results for counting
the number of fixed points\footnote{A fixed point of a graph dynamical
system is a configuration in which no state changes occur.}
for systems with special forms of local functions.
Kosub and Homan \cite{KH-2007} presented dichotomy
results that delineate computationally intractable and efficiently
solvable versions of counting fixed points, based on the class of
allowable local functions.  
The complexity of the predecessor existence problem 
for various classes of underlying graphs and local functions is 
investigated in \cite{BH+07}.
A more general version of the predecessor existence problem,
where the goal is to find $t$-step predecessors for values of $t \geq 2$,
has been studied in \cite{Kawachi-et-al-2017,MR+2018}.
%% for various restricted graph structures (e.g., grid graphs) and for
%% various restricted families of local transition functions
%(e.g., $k$-threshold functions for any $k \geq 2$).
Problems similar to predecessor existence have
also been considered in the context of cellular automata
\cite{Gre-1987,Dur-1994}.
