\section{Introduction}
\label{sec:intro}

\noindent
\textbf{Motivation.}~ \textcolor{red}{(To be added.)}


\smallskip

\noindent
\textbf{Research Questions and Summary of Results.} Informally speaking,
the main question addressed in this paper is whether two configurations
that are similar evolved along similar trajectories.
To formulate a more concrete version of this question, we consider
two similar configurations, say \calcone{} and \calctwo, and
investigate questions regarding the similarity of their predecessor
configurations.
\textcolor{red}{(More to be added.)}


\smallskip

\noindent
\textbf{Related Work.}~
Computational problems associated with 
discrete dynamical systems 
have been addressed by many researchers.
For example, Barrett et al. \cite{BH+06} studied the
reachability problem under the sequential
update model, where a permutation of the vertices is also given,
and state updates are carried out in the order specified by the
permutation.  
%% They established computational intractability results
%% for the general case and obtained polynomial time algorithms when
%% local transition functions are limited to certain Boolean functions.
Bounds on the lengths of transients and cycles in restricted versions
of dynamical systems under the sequential update model are established
in \cite{MR-2007}.  
%% Wooldridge \cite{Woo-2002} presents a discussion
%% of complexity results for multi-agent systems.  
Tosic \cite{Tos-2010,Tosic-2017} presented results for fixed point enumeration
problems for systems with special forms of local transition
functions.  Kosub and Homan \cite{KH-2007} presented dichotomy
results that delineate computationally intractable and efficiently
solvable versions of counting fixed points, based on the class of
allowable local transition functions.  
References \cite{BH+07,MR+2018} considered the predecessor existence problem and
its generalization (called the configuration sequence 
completion problem)  for SyDSs.  
%% They present hardness results
%% for various restricted graph structures (e.g., grid graphs) and for
%% various restricted families of local transition functions
( e.g., $k$-threshold functions for any $k \geq 2$).
Problems similar to predecessor existence have
also been considered in the context of cellular automata
\cite{Gre-1987,Dur-1994}.
