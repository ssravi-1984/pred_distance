\section{Analytical Results}
\label{sec:analysis}

In this section, we present several analytical results regarding the
similarities of predecessor sets of two configurations of a SyDS.
%These results point out that SyDSs may exhibit 
%extreme levels of behavior with respect to similarity.
Throughout this section, the reader should bear in mind that
for any configuration \calc, \predset{\calc}{}
denotes the set of all predecessors of \calc.

Our first result uses the simple observation that if every local function of a SyDS
is the identity function,
then the Hamming distance between  configurations is preserved by the successor function.

%% The second result shows that there are SyDSs such that 
%% there is a pair of distinct configurations that is $1$-close, but 
%% have predecessors which have the maximum level of dissimilarity
%% (i.e., the Hamming distance these predecessor configurations
%% is $n$, the number of nodes).


\begin{proposition}\label{pro:close-close}
Let $G$ be an arbitrary graph.
Then, there is a SyDS $\cals${} with underlying graph $G$,
such that $\cals${} has the following properties: 
(i) every configuration has a predecessor, and (ii) 
for any pair of distinct configurations
\calcone{} and \calctwo{},
$\calh(\successor{\calcone}, \successor{\calctwo)} = \calh(\calcone, \calctwo)$,
and $\maxsep\left( \predset{\calcone},~ \predset{\calctwo} \right)
  =  \calh(\calcone, \calctwo)$.
\end{proposition}

\noindent
\textbf{Proof:}~ The idea is to construct $\cals${} so that each configuration
is a fixed point.
Thus, each configuration \calc{} has a \emph{unique} predecessor, namely \calc{} itself.
As a direct consequence, if configurations \calcone{} and
\calctwo{} are $h$-close,
then their corresponding successors and predecessors are also $h$-close.

Such a SyDS $\cals${} can be constructed as follows.
The underlying graph of $\cals${}  is the given graph $G$.
For each node $v$, the local function is the \emph{identity} function.
In other words, the value of local function $f_v$ is the 
current state value of $v$, regardless of the state values of 
the other nodes in the closed neighborhood of $v$.
In such a SyDS, it can be seen that successor of each configuration \calc{}
is \calc{} itself; that is, each configuration is a fixed point. \QED

Our next result shows that there are SyDSs for which 
there are two distinct configurations that are $1$-close, but
their predecessors are highly dissimilar; that is, they have 
the maximum possible Hamming distance.

\begin{proposition}\label{pro:close-far}
Let $G$ be an arbitrary connected graph, 
and let $n$ be the number of nodes in $G$.
Then, there is a SyDS $\cals${} with underlying graph $G$,
such that $\cals${} has the following properties: 
(i) every configuration has a predecessor, 
(ii) every local function is linear and
(iii) for every configuration
\calcone, there is a configuration \calctwo{} such that 
$\calh(\calcone, \calctwo) = 1$ and 
$\minsep\left( \predset{\calcone},~ \predset{\calctwo} \right)$ ~=~ $n$.
\end{proposition}

\noindent
\textbf{Proof:} 
The underlying graph of $\cals${}  is the given graph $G$.
Before constructing the local functions of $\cals${}, 
we first construct a  spanning tree for $G$.
Then we choose any node $w$ of G, and call this node the {\em root node}.
For every other node $v$,
we consider the spanning tree neighbor of $v$ 
that is on the path from the root node to $v$.
We call this node the {\em key neighbor} of $v$, 
and denote it as $k(v)$.  
For any node $v$,
we let $K(v)$ denote the set of nodes along 
the spanning tree path between the root node and $v$.
Note that $K(v)$ includes the endpoints of this path.

We now specify the local functions of $\cals${}.
The local function of the root node $w$ is the \emph{identity} function;
that is, the value of the local function is the current state value of $w$.
For any other node $v$ the local transition function of $v$ is $v \oplus k(v)$, that is,
the exclusive-or of $v$ and its key neighbor.
This completes the construction of $\cals${}.

For any configuration \calc{} and nonempty set of nodes  $X$,
we let $\hat{\calc}(X)$ denote the exclusive-or of $\calc(u)$ 
over the multiset of values  $\{\calc(u) | \, u \in X\}$.

Consider any configuration \calc.
Let \calcp{} be the configuration where for each node $v$, 
$\calcp(v) = \hat{\calc}(K(v))$.
Let $\calc''$ be the successor of \calcp.
If $v$ is the root node, then $K(v) = \{v\}$, so $\calcp(v) = \calc(v)$.
Since the local function for the root node is the identity function,
$\calc''(v) = \calc(v)$.
If $v$ is not the root node, then from the local function for $v$,
$\calc''(v) = \calcp(v)  \oplus \calcp(k(v))$.
From the construction of \calcp, note that
$\calcp(v) = \hat{\calc}(K(v))$ and $\calcp(k(v))$ =  
$\hat{\calc}(K(k(v)))$.
Further, $K(v) = K(k(v)) \cup \{v\}$.
Thus, $\hat{\calc}(K(v))$ = $\hat{\calc}(K(k(v))) \oplus \calc(v)$.
Hence, $\calc''(v)$ = $\hat{\calc}(K(k(v))) \oplus \calc(v) \oplus  \hat{\calc}(K(k(v)))$
= $\calc(v)$.
Consequently, \calcp{} is a predecessor of \calc.

Since each configuration of $\cals${}  has a predecessor, 
and each configuration has exactly one successor,
each configuration has a unique predecessor.

Now consider any configuration $\calcone$.
Let \calctwo{} be the configuration that differs from 
\calcone{} only in the value of the root node.
We claim that the predecessors $\calcone'$ and $\calctwo'$
of \calcone{} and \calctwo{}, respectively, 
differ in the values of all nodes.

If $v$ is the root node, then since \calcone{} and \calctwo{} 
differ in the value of the root node,
$\calcone'$ and $\calctwo'$ also differ in the value of the root node.

Suppose that $v$ is not the root node.
Then $\calcone'(v) = \hat{\calcone}(K(v))$ and $\calctwo'(v) = \hat{\calctwo}(K(v))$.
The values in $K(v)$ in \calcone{} and \calctwo{} differ 
in the value of the root node, and no other node.
Thus, $\calcone'(v)$ and $\calctwo'(v)$ have complementary values. \QED

\smallskip
We now show the existence of SyDSs in which there 
are pairs of configurations which have the maximum level 
of dissimilarity but their predecessors are 1-close. 

\begin{proposition}\label{pro:far-close-degree}
Let $G$ be an arbitrary graph, 
and let $\Delta$ be the the maximum node degree of $G$.
Then, there is a SyDS $\cals${} with underlying graph $G$,
such that $\cals${} has the following properties: 
(i) every configuration has a predecessor, 
(ii) every local function is linear and
(iii) for every configuration
\calcone{}, there is a configuration \calctwo{} such that 
 $\calh(\calcone, \calctwo) = \Delta + 1$  and 
$\maxsep\left( \predset{\calcone},~ \predset{\calctwo} \right)$ ~=~ $1$.
\end{proposition}

\noindent
\textbf{Proof:}~ 
The underlying graph of $\cals${}  is the given graph $G$.
We select a node $v_1$ of $G$ with node degree $\Delta$.
Let $V_1$ denote the node set consisting of $v_1$ and all nodes that are not neighbors of $v_1$.
Let $V_2$ denote the remaining set of nodes, i.e. the $\Delta$ neighbors of $v_1$.

We make the local transition function of nodes in $V_1$ be the identity function.  
We make the local transition function of each node $u$ in $V_2$ 
be the exclusive or of $u$ and $v_1$.
This completes the construction of $\cals${}.

Let \calcone{} be an arbitrary configuration of $\cals${}.
Let \calctwo{} be the configuration such that nodes $v_1$ and its $\Delta$ neighbors
have the complements of their values in \calcone{},
and all other nodes have the same value as in \calcone{}.
Thus, $\calh(\calcone, \calctwo) = \Delta + 1$.
At this point, we can assume wlog that $\calcone(v_1) = 0$ and $\calctwo(v_1) = 1$.


It can be seen that \calcone{} is a fixed point of $\cals${},
so \calcone{} is a predecessor of itself.
Let $\calcone' = \calcone$.
Let $\calctwo'$ be identical to $\calcone'{}$, except that $\calctwo(v_1) = 1$.
Thus, $\calh(\calcone', \calctwo') =1$.
It can be seen that the successor of $\calctwo'$ is \calctwo{},
so $\calctwo'$ is a predecessor of \calctwo{}.
Thus, $\maxsep\left( \predset{\calcone},~ \predset{\calctwo} \right)$ ~=~ $1$.

Observe that we have also shown that every configuration has a predecessor.
\QED


%%
%% \begin{wrapfigure}[9]{l}{0.35\textwidth}
%% \centering
%% \input{star_graph.pdf_t}
%% \caption{\small{Star graph used in the proof of
%% Proposition~\ref{pro:far-close}.}}
%% \label{fig:star-graph}
%% \smallskip
%% \end{wrapfigure}
%%

\smallskip

\begin{proposition}\label{pro:far-close}
For any integer $n \geq 2$,
there is a SyDS $\cals${} with $n$
nodes satisfying the following properties: 
(i) there is a pair of configurations
\calcone{} and \calctwo{} with $\calh(\calcone, \calctwo) = n$  and 
$\maxsep\left( \predset{\calcone},~ \predset{\calctwo} \right)$ ~=~ $1$.
\end{proposition}

\noindent
\textbf{Proof:}~ 
Let $G$ be the star graph on $n$ nodes.
%% as shown in Figure~\ref{fig:star-graph}.
The result follows from Proposition \ref{pro:far-close-degree}.
\QED


%%
%% The underlying graph of the SyDS $\cals_3$ is
%% the star graph on $n$ nodes shown in Figure~\ref{fig:star-graph}.
%% The local function at the root node $v_1$ is the \emph{identity}
%% function.
%% For each node $v_i$, where $2 \leq i \leq n$, the local function
%% $f_i$ is the exclusive or of the state of $v_i$ and that of $v_1$.  
%% This completes the specification of $\cals_3$.

%% Let \calcone{} be the configuration where each node is in state 1.
%% Similarly, let \calctwo{} be the configuration where each node is in state 0.
%% Thus, $\calh(\calcone, \calctwo) ~=~ n$.
%% It can be verified that the configuration $\calcone'$, 
%% where the state of $v_1$ is 1 and the states of all other
%% nodes are 0 is a predecessor of \calcone.
%% We claim that $\calcone'$ is the \emph{only} predecessor of \calcone.
%% To see this, note that  in any predecessor of \calcone, 
%% the state of node $v_1$ must be 1
%% since the local function at $v_1$ is the 
%% identity function and $\calcone(v_1) = 1$.
%% Now, for any node $v_i$, $2 \leq i \leq n$, 
%% suppose the state of $v_i$ in the predecessor is 1. 
%% Then, since the local function at $v_i$ is the exclusive or of the states of $v_1$
%% and $v_i$, the next state of $v_i$ will be 0 where as \calcone{} specifies
%% the state of $v_i$ as 1. 
%% Therefore, in any predecessor, for $2 \leq i \leq n$, the state of $v_i$ must be 0.
%% Hence, \calcone{} has a \emph{unique} predecessor, namely $\calcone'$.
%% In a similar manner, it can be seen that \calctwo{} has a unique predecessor 
%% $\calctwo'$, which is \calctwo{} itself.
%% Thus, $\calh(\calcone', \calctwo') ~=~ 1$.
%% Since the predecessors of \calcone{} and \calctwo{} are unique,
%% it follows that $\maxsep(\Pi(\calcone), \Pi(\calctwo)) = 1$, as indicated
%% in the statement of Proposition~\ref{pro:far-close}. \QED

\smallskip

\begin{proposition}\label{pro:far-close-lower-bound}
Suppose that for a given SyDS $\cals${},
the maximum node degree of the underlying graph is $\Delta$.
 Then, for any pair of configurations
\calcone{} and \calctwo{},
 $\calh(\Sigma(\calcone), \Sigma(\calctwo)) \leq (\Delta+1) \calh(\calcone, \calctwo)$.\end{proposition}

\noindent
\textbf{Proof:}~ 
The value of a given node in a configuration can only affect the values of at most $\Delta +1$
nodes in the successor of that configuration.
\QED





\newcommand{\mps}{\mbox{MPS}}
\newcommand{\mpsgp}{\mbox{MPSGP}}
\newcommand{\pre}{\mbox{PRE}}

\smallskip

Finally, we present a simple result that establishes the computational complexity
of computing distance measures for predecessor configurations.
The decision problem, which we call \textbf{Minimum Predecessor Separation} (\mps),
is the following: given a SyDS \cals{}, two configurations $\calcone{}$ and
$\calctwo{}$, and a positive integer $q$, is the value of
$\minsep(\Pi(\calcone), \Pi(\calctwo))$ at most $q$?
We have the following result.

\begin{proposition}\label{pro:minsep-hard}
The \mps{} problem is \cnp-complete.
\end{proposition}

\noindent
\textbf{Proof:}~ It can be seen that \mps{} is in \cnp.
To prove \cnp-hardness,
we use a reduction from the 
\textbf{Predecessor Existence} (\pre) problem: given a SyDS \cals{} and
a configuration \calc, does \calc{} have a predecessor?
The \pre{} problem was shown to be \cnp-hard in \cite{BH+07},
even for SyDSs in which each node computes the 2-threshold function.
We will use this problem in our reduction.

\smallskip

\noindent
The reduction from the special version of the \pre{} problem to \mps{}
is simple.
\begin{enumerate}
\item The SyDS \cals{} in the \mps{} instance is the same as the SyDS in the
\pre{} instance.
\item We set configuration \calcone{} of the \mps{} to be the one in which
every node is in state 0.
The configuration \calctwo{} of the \mps{} is the same as the configuration
\calc{} specified in the \pre{} instance.
(In configuration \calc, some nodes have the value 1. So,
\calc{} which is the same as \calctwo, is different from \calcone.)
\item Let $n$ be the number of nodes in \cals. We set $q = n$.
\end{enumerate}
This completes the reduction. It can be seen that the reduction
can be carried out in polynomial time.

Suppose the \pre{} problem instance has a solution.
Let $\calcp$ be a predecessor of \calc.
Since each local function of \cals{} is the 2-threshold function,
configuration \calcone{} (where each node is in state 0) is a fixed point.
Thus, $\Pi(\calcone)$ is nonempty; it includes \calcone.
Since \calctwo{} is the same as \calc{} and \calc{} has a
predecessor \calcp,  $\Pi(\calctwo)$ is also nonempty. 
Hence, it follows that $\calh(\Pi(\calcone), \Pi(\calctwo))$ is at most $n$.
In other words, the \mps{} instance has a solution.

Now suppose the \mps{} instance has a solution.
Since the bound $q$ on the value of 
$\minsep(\Pi(\calcone), \Pi(\calctwo))$ is finite, 
it follows that both 
$\Pi(\calcone)$ and $\Pi(\calctwo))$ are nonempty. 
In other words, the configuration \calctwo{} = \calc{}
has a predecessor; that is, the \pre{} instance has a solution.
This completes the proof of Proposition~\ref{pro:minsep-hard}. 
\QED

We now show that the MDS problem is \cnp-complete
even when given predecessors of \calcone{} and \calctwo{}.
We call the decision problem when this extra information is given
\textbf{Minimum Predecessor Separation Given Predecessors} (\mpsgp).
Note that since the predecessors of \calcone{} and \calctwo{}
are specified in a given \mpsgp{} problem instance, 
it is unnecessarily to explicitly specify \calcone{} and \calctwo{}.
Thus, we formalize the \mpsgp{} problem as follows:
 given a SyDS \cals{}, and two configurations $\calcone{} '$ and $\calctwo{}'$,
is 
$\minsep(\Pi(\Sigma(\calcone')), \Pi(\Sigma(\calctwo'))) < \calh(\calcone', \calctwo')$?

\begin{proposition}\label{pro:minsep-gp-hard}
The \mpsgp{} problem is \cnp-complete.
\end{proposition}

\noindent
\textbf{Proof:}~ 
 It can be seen that \mpsgp{} is in \cnp.
To prove \cnp-hardness,
we use a reduction from 3SAT.
Suppose the given 3SAT problem instance $\psi$ involves $n$ variables and $m$ clauses.
We can assume wlog that $m \geq 2$.
The  \mpsgp{} problem instance is constructed as follows.
\begin{enumerate}
\item The underlying graph of SyDS \cals{} has $n+m+1$ nodes: a central node $z$,
a node $v_i$ for each variable of $\psi$,
and a node $c_j$ for each clause of $\psi$.
\item The underlying graph of SyDS \cals{} 
has an edge between $z$ and every other node,
and an edge between each clause node $c_j$ 
and the nodes for the variables occurring in that clause.
\item The local function for node $z$ is 
the OR of $z$ and the AND of the complements of all the clause nodes.
\item The local function for a variable node $v_i$, $1 \leq i \leq n$, is 
 $z \vee \overline{v_i}$.
\item The local function for a clause node $c_j$, $1 \leq j \leq m$, is 
 the function that is 1 iff either $z$ and $c_j$ are both 1
  and the values of the variable nodes adjacent to node $c_j$
 satisfy clause $c_j$ of $\psi$,
 or $z$ and $c_j$ are both 0.
\item We set configuration $\calcone{}'$ of the \mpsgp{} to be the one in which
every node has value 0.
\item We set configuration $\calctwo{}'$ of the \mpsgp{} to be the one in which
node $z$ has value 0 and every other node has value 1.
\end{enumerate}

This completes the reduction. It can be seen that the reduction
can be carried out in polynomial time.
Also note that if in a given configuration node $z$ has value 0, 
then all the nodes for the variables and clauses 
are complemented in the successor configuration.

Given the constructed \mpsgp{} problem instance,
let $\calcone = \Sigma(\calcone')$, and $\calctwo = \Sigma(\calctwo')$.
Note that in \calcone{} all nodes have value 1,
and in \calctwo{} all nodes have value 0.
Also, note that \calctwo{} is identical to $\calcone{}'$.

Suppose that $\psi$ is satisfiable.
Let $\alpha$ be a satisfying assignment.
Let $\calc_{\alpha}$ be the configuration of  \cals{}
where node $z$ has the value 1,
the variable nodes have the same values as in $\alpha$,
and the clause nodes all have value 1.
It can be seen that $\Sigma(\calc_{\alpha})$ is \calcone{},
so $\calc_{\alpha}$ is a predecessor of \calcone{}.
Note that $\calh(\calcone', \calctwo') = n+m$.
Also, note that in $\calc_{\alpha}$, at least two nodes have value 1.
Thus $\calh(\calc_{\alpha}, \calctwo') < n+m  = \calh(\calcone', \calctwo')$.
Consequently, 
$\minsep(\calcone, \calctwo) < \calh(\calcone', \calctwo')$,
so the answer to the constructed \mpsgp{} instance is YES.

Now suppose that the answer to the constructed \mpsgp{} instance is YES.

Let $\calctwo''$ be a predecessor of \calctwo{}.
Since $\calctwo(z) = 0$, the local function of node $z$ requires that 
$\calctwo''(z) = 0$.
The local function of every other node then requires that 
every other node has value 1 in  $\calctwo''$.
Thus,  $\calctwo'' = \calctwo'$,
so $\calctwo'$ is the only predecessor of $\calctwo$.
Since the answer to the constructed \mpsgp{} instance is YES,
there is  a predecessor $\calcone''$ of \calcone{} such that
$\calh(\calcone'', \calctwo') < \calh(\calcone', \calctwo')$.


Recall that in \calcone{}, all nodes have value 1.
From the local transition functions of \cals{}, 
the only predecessor of \calcone{} for which node $z$ has value 0 is $\calcone{}'$.
Thus, predecessor $\calcone''$ of \calcone{}
must have $\calcone''(z) = 1$.
Consider a given clause node, say $c_j$.
Since $\calcone(c_j) = 1$ and $\calcone''(z) = 1$,
the transition function for node $c_j$ requires that
the values of the variable nodes in $\calcone''$ satisfy clause $c_j$.
Since this requirement holds for each clause node,
the values of the variable nodes in $\calcone''$ satisfy $\psi$.

Thus $\psi$ is satisfiable iff the answer to the constructed \mpsgp{} instance is YES.
This completes the proof of the reduction.
\QED