\section{Analytical Results}
\label{sec:analysis}

In this section, we present several analytical results regarding the
similarities of predecessor sets of two configurations of a SyDS.

The first result shows that there are SyDSs such that 
if a pair of distinct configurations is $h$-close, then 
any pair of their predecessors is also $h$-close.
Recall that for any configuration \calc, \predset{\calc}{}
denotes the set of all predecessors of \calc.


\begin{proposition}\label{pro:close-close}
For integers $n$ and $h$, where $n \geq 1$ and $1 \leq h \leq n$,
there is a SyDS $\cals_1${} with $n$
nodes satisfying both of the following properties: 
(i) every configuration has a predecessor and (ii) 
for any pair of distinct configurations
\calcone{} and \calctwo{} that are $h$-close, 
$\maxsep\left( \predset{\calcone},~ \predset{\calctwo} \right)$ ~=~ $h$.
\end{proposition}

\noindent
\textbf{Proof:}~ The idea is to construct a SyDS in which each configuration
is a fixed point.
Thus, each configuration \calc{} has a predecessor, namely \calc{} itself.
As a direct consequence, we have that if configurations \calcone{} and
\calctwo{} are $h$-close for some integer $h$, $1 \leq h \leq n$,
then the corresponding predecessors are also $h$-close.

Such a SyDS $\cals_1${} can be constructed as follows.
The underlying graph $G(V,E)$ has $n$ nodes and the set of edges is arbitrary.
For each node $v \in V$, the value of local function $f_v$ is the 
current state value of $v$; thus, $f_v$ ignores the states of all the nodes
to which $v$ has an edge in $G$.
In such a SyDS, it can be seen that successor of each configuration \calc{}
is \calc{} itself; that is, each configuration is a fixed point. \QED

Our next result shows that there are SyDSs for which 
there is a pair of distinct configurations is $1$-close, but
their predecessors are highly dissimilar; that is, they have 
the maximum possible Hamming distance.

\begin{proposition}\label{pro:close-far}
For any integer $n \geq 1$,
there is a SyDS $\cals_2${} with $n$
nodes satisfying both of the following properties: 
(i) every configuration has a predecessor and (ii) 
there is a pair of configurations
\calcone{} and \calctwo{} that are $1$-close but 
$\minsep\left( \predset{\calcone},~ \predset{\calctwo} \right)$ ~=~ $n$.
\end{proposition}

\noindent
\textbf{Proof:} \textcolor{red}{(To be checked.)}

Consider the following SyDS $\cals_2$.
The underlying graph $G(V,E)$ is a simple cycle on $n$ nodes.
Choose any node of $G$, and call this node $x$ the {\em key node}. For
every other node $v$, choose any neighbor $w$ of $v$ that is closer to the
key node than $v$. Call this node $w$ the {\em key neighbor} of $v$.
The local transition function of the key node $x$ is the identity
function (i.e., the value of the local function $f_x$ at $x$ is
the current state of $x$).
For any other node $v$, the local transition function of
$v$ is the XOR (i.e., exclusive or) of the state of $v$ 
and that if its key neighbor.
This completes the construction of S.

Now consider any configuration \calcone. 
Let \calctwo{} be the configuration that differs 
from \calcone{} only in the value of the key node.

We claim that \calcone{} and \calctwo{} have predecessors that differ in the
values of all nodes. 
Let $\calcone'$ and $\calctwo'$ denote the predecessors of
\calcone{} and \calctwo.
We use induction on the distance of a given node from the key node.
First, since the transition function of the key node is the identity function,
$\calcone'$ and $\calctwo'$ differ on the key node.
Now, consider any nonkey node $v$.  Suppose its key neighbor is $u$.
From the induction hypotheses, $\calcone'$ and $\calctwo'$ differ on $u$.
Since \calcone{} and \calctwo{} have the same value of $v$, 
but their predecessors have different values of $u$, 
$\calcone'$ and $\calctwo'$ have different values of $v$. \QED
