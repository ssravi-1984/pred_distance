%%\vspace*{-.30in}
\section{Preliminaries}
\label{sec:prelim}

%%\vspace*{-.22in}
\textbf{Synchronous Dynamical Systems and Local Functions.}~
We follow the presentation in \cite{BH+07} to discuss
the basic definitions associated with discrete dynamical systems.
Let \bbb{} denote the Boolean domain \{0,1\}.
A \textbf{Synchronous Dynamical System} (SyDS)
\cals{} over \bbb{} is specified as
a pair ${\cal S}  = (G, {\cal F})$, where 
(a)~ $G(V,E)$, an undirected graph with $|V| = n$, 
represents the underlying graph of the SyDS,
with node set~$V$ and edge set~$E$, and 
(b)~ ${\cal F} = \{f_1, f_2, \ldots, f_n \}$ is a collection of
functions in the system, with
$f_i$ denoting the \textbf{local function} 
associated with node $v_i$, $1 \leq i \leq n$.
Each node of $G$ has a state value from \bbb. 
Each function $f_i$ specifies the local interaction
between node $v_i$ and its neighbors in $G$.
The inputs to function $f_i$ are the state of $v_i$ and
those of the neighbors of $v_i$ in $G$; function $f_i$ maps 
each combination of inputs to a value in \bbb.
This value becomes the next state of node $v_i$. 
It is assumed that each local function can be computed efficiently.
%% We will provide examples of local functions shortly.

%%\medskip
At any time $\tau$, 
the {\bf configuration} \calc{} of a SyDS 
is the $n$-vector $(s_1^{\tau}, s_2^{\tau}, \ldots, s_n^{\tau})$,
where $s_i^{\tau} \in \bbb$ is the state of
node $v_i$ at time $\tau$ ($1 \leq i \leq n$).
Given a configuration \calc, the state of a node $v$ in \calc{}
is denoted by $\calc(v)$.
In a SyDS, all nodes compute and update their next state  
\emph{synchronously}.
Other update disciplines (e.g., sequential updates) 
have also been considered in the literature (e.g., \cite{BH+07}).
Suppose a given SyDS transitions in one step from
a configuration \calcp{} to a configuration \calc.
Then we say that \calc{} is the \textbf{successor} of \calcp,
and \calcp{} is a \textbf{predecessor} of \calc.
Since the SyDSs considered in this paper are deterministic,
each configuration has a unique successor.
However, a configuration may have zero
or more predecessors.
In the graph dynamical systems literature, configurations with 
no predecessors are called \textbf{Garden of Eden}
(GE) configurations \cite{MR-2007}.
SyDSs have been considered in the literature under many 
classes of local interaction functions 
(see e.g., \cite{BH+11,Kawachi-et-al-2017}).
We now present an an example of a SyDS where the local function
at each node is a \textbf{threshold function}. 
For each integer $k \geq 0$, the $k$-\textbf{threshold function} has the
value 1 iff at at least $k$ of its inputs are 1.

\smallskip

%%
\begin{wrapfigure}[16]{l}{0.35\textwidth}
\centering
\input{syds_example.pdf_t}
\caption{\small{An Example of a SyDS where each node has a
threshold function. The threshold values are shown in parentheses.}}
\label{fig:syds-example}
\smallskip
\end{wrapfigure}
%%

%%
\noindent
\textbf{Example:}~
The underlying graph of a SyDS shown in
Figure~\ref{fig:syds-example}.
The threshold value for each node is shown within parentheses. 
(Thus, the local function at $v_2$ is the 2-threshold function
while that at $v_4$ is the 3-threshold function.)
Suppose the initial configuration of the system is $(1, 1, 0, 0, 0)$;
that is,
$v_1$ and $v_2$ are in state 1 while 
$v_3$, $v_4$ and $v_5$ are in state 0. 
At time 1, the state of $v_3$ changes to 1 (since its threshold is 1
and the state of its neighbor $v_2$ is 1) while
the other nodes retain their state values.
At time 2, the state of $v_4$ changes to 1 
while the other nodes retain their state values.
At time 3, the state of $v_5$ also changes to 1,
and the system reaches the configuration $(1, 1, 1, 1, 1)$.
No further state changes occur in subsequent time steps;
that is, the configuration $(1, 1, 1, 1, 1)$ is a \textbf{fixed point}.

\smallskip


\smallskip

\noindent
\textbf{Additional Terminology.}~ Here, we introduce some additional
graph theoretic and Boolean function terminology that will be used 
throughout the paper. 
Given a graph $G(V,E)$ and a node $v \in V$, the \textbf{closed neighborhood}
of $v$, denoted by $N[v]$, consists of $v$ and all its neighbors.
%% The definitions of \emph{tree decomposition} and
%% \emph{treewidth} are standard; they can found in
%% many references (e.g., \cite{Bod97}).
%% We also need the definition of symmetric and $r$-symmetric Boolean functions;
%% these definitions are from \cite{Crama-Hammer-2011} 
%% and \cite{BH+07} respectively.
%% \begin{definition}\label{def:sym_funct}
A \textbf{symmetric} Boolean function \cite{Crama-Hammer-2011}  is one whose
value does not depend on the order in
which the input bits are specified;
that is, the function value depends only on how many
of its inputs are 1.
It can be seen that for any $k$, $k$-threshold functions belong to the 
class of symmetric functions.
A Boolean function $f$ is \textbf{$r$-symmetric} \cite{BH+07}
if the inputs to $f$ can be partitioned into at most $r$ classes
such that the value of $f$ depends only on how many of the
inputs in each of the $r$ classes are 1.
Note that any symmetric function is $1$-symmetric.
%% \end{definition}
Also, any Boolean function with $d$ inputs is $d$-symmetric.
We will be mainly concerned with $r$-symmetric functions 
where $r$ is a \emph{fixed} integer.
%% We say that a \underline{SyDS is $r$-symmetric}
%% if each of its local functions is $r'$-symmetric 
%% for some $r' \leq r$.

\smallskip

\noindent
\textbf{Hamming Distance and Similarity of Configurations.}~
Given two configurations \calcone{} and \calctwo{} of a SyDS
over the domain $\{0,1\}$,
the \textbf{Hamming Distance} between \calcone{} and \calctwo, 
denoted by \calh(\calcone, \calctwo), is
the number of positions in which they differ.
For example, if
\calcone{} = (1,0,0,1) and \calctwo{} = (0,1,0,0),   
then \calh(\calcone, \calctwo) = 3.  
We say that two configurations \calcone{} and \calctwo{} 
of a SyDS are $h$-\textbf{close} if \calh(\calco, \calct) ~=~ $h$. 
Two configurations that are $h$-close for a small value of $h$
can be thought of as `similar' configurations.  
We note that in a SyDS with $n$ nodes, 
the maximum Hamming distance between any pair of configurations
\calcone{} and \calctwo{} is $n$; this occurs when \calcone{}
is the bitwise complement of \calctwo. 

\smallskip

\noindent
\textbf{Similarity Measures for Sets of Configurations.} Recall
that our focus is on studying the similarity between predecessors
of similar configurations.
To do this, we define several distance measures between two sets of
configurations.
Suppose $S_1$ and $S_2$ are two nonempty sets of configurations.
We will consider the following distance measures.

\newcommand{\minsep}{\mbox{\textsc{MinSep}}}
\newcommand{\maxsep}{\mbox{\textsc{MaxSep}}}
\newcommand{\avgsep}{\mbox{\textsc{AvgSep}}}

\begin{description}
\item{(a)} \textbf{Minimum Separation} (\minsep): This measure is defined as follows:
\[ 
\minsep(S_1, S_2) ~=~ \min\{ \calh(\calc, \calcp) ~:~ 
                                \calc{} \in S_1,~ \calcp{} \in S_2\}.
\]
\item{(b)} \textbf{Maximum Separation} (\maxsep): This measure, which is
analogous to minimum separation, is defined as follows.
\[ 
\maxsep(S_1, S_2) ~=~ \max\{ \calh(\calc, \calcp) ~:~ 
                                \calc{} \in S_1,~ \calcp{} \in S_2\}.
\]
\item{(c)} \textbf{Average Separation} (\avgsep): This measure is defined as follows.
\[ 
\avgsep(S_1, S_2) ~=~ \displaystyle{\frac{\sum_{\mathcal{C} \in S_1,~ \mathcal{C}' \in S_2}
                            \calh(\calc, \calcp)}{|S_1| \times |S_2|}}.
\]
\end{description}
Among the above measures, a small value of \maxsep{} provides the 
strongest guarantee of similarity.
This is because if \maxsep($S_1$, $S_2$) = $\alpha$, and $\alpha$ is small,
then the Hamming distance
between any pair configurations \calc{} and \calcp, where \calc{} $\in S_1$
and \calcp{} $\in S_2$, is at most $\alpha$; in other words, 
each such configuration pair is $\alpha$-close.

\smallskip

In this paper, we will consider the above measures for pairs of sets
which are \emph{disjoint}.
For convenience, when one or both of the sets of configurations
$S_1$ and $S_2$ is empty, we will define the values of 
$\minsep(S_1, S_2)$,
$\maxsep(S_1, S_2)$ and $\avgsep(S_1, S_2)$ to be $\infty$.

%% which are nonempty and \emph{disjoint}. 
%% Thus, the values of all the measures will be \emph{strictly positive}. 

\smallskip

Given a configuration \calc{} of a SyDS, we will use the notation \predset{\calc}{}
to denote the set of all predecessors of \calc.
The following lemma points out a simple property of predecessor sets
of SyDSs.

\begin{lemma}\label{lem:disjoint_pred}
Let \cals{} be a SyDS and let \calcone{} and \calctwo{} be two different
configurations of \cals.
The sets \predset{\calcone}{} and \predset{\calctwo}{} are disjoint.
\end{lemma}

\noindent
\textbf{Proof:}~ The proof is by contradiction.
Suppose \predset{\calcone}{} and \predset{\calctwo}{} are not disjoint.
Let $X$ be a configuration that appears in both
\predset{\calcone}{} and \predset{\calctwo}.
Then $X$ has two different successors, namely \calcone{} and \calctwo.
This is a contradiction since each configuration in a SyDS has a \emph{unique}
successor. The lemma follows. \QED

